\documentclass[10pt]{article}

\usepackage{amsmath}
\usepackage{amssymb}
\usepackage{amsthm}
\usepackage[english]{babel}
\usepackage{graphicx} %figures 
\usepackage{csquotes} %recommended package with bib LaTeX
% \usepackage[style=alphabetic, sorting=ynt]{biblatex} 
\usepackage{faktor} %nicer quotient 
\usepackage{tabularx} %fancy tables
\usepackage{stmaryrd} % lightning symbol 
\usepackage{subcaption} % multiple figures in the same picture
\usepackage{hyperref} % nicer hyperlinks in the text
\usepackage{bbm} %mathbbm 1 command
\usepackage{enumerate} %stylizing enumerate

\newcommand{\bbZ}{\mathbb{Z}}
\newcommand{\bbR}{\mathbb{R}}
\newcommand{\bbF}{\mathbb{F}}
\newcommand{\bbS}{\mathbb{S}}

\newcommand{\ccH}{\mathcal{H}}
\newcommand{\ccO}{\mathcal{O}}

\newcommand{\hmaps}[3][\bbZ_2]{\left[#2, #3\right]_{#1}}
\newcommand{\ztwo}{\bbZ_2}
\newcommand{\hsimp}[3][2]{\mbox{hSimp}_{#1}\left(#2, #3\right)}
\newcommand{\simp}[3][2]{\mbox{Simp}_{#1}\left(#2, #3\right)}
\newcommand{\rar}{\rightarrow}

\newcommand{\scalar}[2]{\left( #1 \vert #2\right)}
\newcommand{\matrice}[1]{\begin{pmatrix}#1\end{pmatrix}}
\newcommand{\norm}[1]{ \lVert #1 \rVert}
\newcommand{\gen}[1]{ \left\langle #1 \right\rangle}

\newtheorem{theorem}{Theorem}
\newtheorem{lemma}{Lemma}
\newtheorem{question}{Question}
\newtheorem{definition}{Definition}

%change \varepsilon to be he default choice
\let\uglyepsilon\epsilon
\let\epsilon\varepsilon



\title{Splitting a measure along a fixed direction}
\addbibresource{./ref.bib}
\author{}

\begin{document}
\maketitle
\section*{Notation}
\begin{itemize}
\item A measure $\mu$ on $\bbR^d$ is nice if it is Borel, $\mu(\bbR^d)=1$, $\mu(H) = 0$ for all affine hyperplanes and it has connected support.
\item Given an arrangement $\ccH$ of $k$ oriented affine hyperplanes and a sign pattern $\alpha\in (\ztwo)^k$, the orthant $\ccO(\ccH, \alpha)$ is defined as
  \[
    \ccO(\ccH, \alpha) := \{x\in \bbR^d \vert (-1)^{\alpha_i}(\scalar{h_i}{x} - a_i) > 0 \}
  \]
  Intuitively, $\ccO(\ccH, \alpha)$ is the set of point lying on the side of $H_i$ determined by $\alpha_i$ (positive if $\alpha_i=0$, negative otherwise).
\item $G= \bbZ_4\times \ztwo$ with generators $g_1 = (1,0)$ and $g_2=(0, 1)$.
\item If $v\in \bbR^d$ is a non-zero vector, $\overline{v}$ is its normalization $\overline{v}=\frac{v}{\norm{v}}$.
\item An oriented affine hyperplane $H = \{x\in \bbR^d \vert \scalar{h}{x} - a =0 \}$ defines a point $\frac{(h, a)}{\norm{(h,a)}}$ in $S^d$; likewise, a point on $x = (\bar x, a)\in S^d$ defines an oriented affine hyperplane $H(x) = \{p\in \bbR^d \vert \scalar{\bar x}{p} -a = 0 \}$.
\item Given a nice measure $\mu$ in $\bbR^3$, define the standard test map $F:\left(S^3 \right)^3 \times \ztwo^3  \rightarrow \bbR$:
  \[
    F(x, \omega) = \mu(\ccO(x, \omega))  - \frac{1}{8}
  \]
  $x$ is an equipartition $\iff$ $F(x, \alpha) = 0$ $\forall \alpha$.
\item Given a test map $F$, the alternating sum functions are the discrete Fourier transform of the test map. More precisely, if $\alpha \in \left( \ztwo\right)^3$
  $\alpha \neq 0$, then the alternating sum with parameter $\alpha$ is:
  \[
    F_\alpha(x) = \sum_{\omega\in \ztwo^3} (-1)^{\scalar{\alpha}{\omega}} F(x, \omega)
  \]
  $x$ is an equipartition $\iff$ $F_\alpha(x) = 0$ $\forall \alpha \neq 0$
\item The group $\ztwo^3$ acts on $\left(S^3 \right)^3$ by coordinate-wise antipodality and, given $\alpha, g \in \ztwo^3$ with $\alpha\neq 0$ and a test map $F$:
  \[
    F_\alpha(g\cdot x) = (-1)^{\scalar{\alpha}{g}}F_\alpha(x)
  \]
\end{itemize}

\section*{The Result}
Our goal is to prove the following result:
\begin{theorem}
  Given a nice measure $\mu$ and a direction $p\in S^2$ it is possible to find $3$ affine oriented planes that equipartition the measure and the
  first two have the prescribed oriented intersection $p$. Formally, $\exists \ccH=((h_1, a_1), (h_2, a_2), (h_3, a_3))$ configuration of oriented planes such that:
  \begin{itemize}
  \item $\forall \alpha\in (\ztwo)^k$, $\ \mu(\ccO(\ccH, \alpha))= \frac{1}{8}$.
  \item $\overline{h}_1\wedge \overline{h}_2 = p$.
  \end{itemize}
\end{theorem}

\begin{proof}
  Without loss of generality, let $z = (0,0,1)$. The proof will be split in two sections: the first deals with constructing a map
  $\Phi: S^1\times S^3 \rightarrow \bbR^4$ in $\bbR^4$ whose zeros codify equipartitions of the mass. The second step will be to show that the map will
  respect a suitable action of $G$ on the two space and that such equivariance forces the existence of a zero.

  \vskip .5em
  \textbf{Step 1}
  \vskip .5em
  The key step in constructing the map is to show that we can parametrize pair of planes that have intersection direction
  $z$ and split the mass in 4 equal parts
  with vector in $S^1$.

  Project the mass on the $xy$ plane to obtain a nice measure $\mu^\#$ on $\bbR^2$. The following bisecting lemma applies:


  \begin{lemma}[Bisecting]
    Let $\mu^\#$ be a nice measure on $\bbR^2$ and $v\in S^1$ a direction. Then there exists two oriented affine lines $l_0 =\bbR\vec{l_0} + a_0$ and $l_1 = \bbR\vec{l_1} + a_1$ in $\bbR^2$ such that:
    \begin{itemize}
    \item $l_0$ and $l_1$ equipartition $\mu^\#$
    \item $v$ bisects the angle between $l_0$ and $l_1$
    \end{itemize}

    What is more, we can choose consistently the direction $\vec{l_0}$ (e.g. fix $\vec{l_0}$ to be the first direction clockwise, while the first one contraclockwise is $\vec{l_1}$).
    Once this choice is made, $\vec{l_0}$ and $\vec{l_1}$ are unique and depend continuously on $v$.
  \end{lemma}

  The lemma guarantees that, once we fix a direction $v\in S^1\subseteq S^2$ (inclusion as the horizontal equator in $S^2$)
  there are two affine lines in the $xy$ plane $l_0 = \bbR\vec{l_0}(v) + a_0(v))$ and $l_1 = \bbR\vec{l_1}(v) + a_1(v))$ that bisect the projected measure $\mu^\#$.

  Define $H_i(v) = (h_i(v), a_i(v))$ to be the affine (oriented) span of $l_i(v)$ and $z$, the two planes now equipartition the measure $\mu$ and have the desired intersection.

  Let now $\bbZ_4 = \gen{g_1}$ act on $S^1$ by $\frac{\pi}{2}$ rotation counter clockwise. Then, by construction we have that:
  \[
    H_0(g_1\cdot v) = H_1(v)
  \]
  \[
    H_1(g_1\cdot v) = -H_0(v)
  \]

  In other words, if we consider the planar problem with the bisecting vector rotated by $\frac{\pi}{2}$, the new first line is the previous second while the new second is
  is the old first with opposite orientation.

  We can now define a map $S^1\times S^3 \rightarrow (S^3)^3$, $(v, w) \mapsto (H_0(v), H_1(v), w)$.

  To check now that a configuration of $3$ planes equipartition the measure, it is equivalent to check that the alternating sum functions $F_\alpha$ are $0$ for all signed
  patterns $\alpha\in \ztwo^3 \setminus 0$. However, by construction the first two planes split equally the mass, thus it is sufficient to check the $4$
  patterns involving the last one; i.e. $\alpha = (0,0,1), (1,0,1), (0,1,1)$ and $(1,1,1)$.

  What is more, it is straightforward  how the alternating sum functions behave under the $G$-action.
  Explicitly, $\forall (v,w)\in S^1\times S^3$ the following equalities hold:

  \begin{align*}
    F_{(0,0,1)}(g_1\cdot (v,w)) &=  F_{(0,0,1)}(v,w)\\
    F_{(0,1,1)}(g_1\cdot (v,w)) &=   - F_{(1,0,1)}(v,w)\\
    F_{(1,0,1)}(g_1\cdot (v,w)) &=  F_{(0,1,1)}(v,w)\\
    F_{(1,1,1)}(g_1\cdot (v,w)) &=  - F_{(1,1,1)}(v,w)\\
  \end{align*}

  \begin{align*}
    F_{(0,0,1)}(g_2\cdot (v,w)) &=  - F_{(0,0,1)}(v,w)\\
    F_{(0,1,1)}(g_2\cdot (v,w)) &=  - F_{(1,0,1)}(v,w)\\
    F_{(1,0,1)}(g_2\cdot (v,w)) &=  - F_{(0,1,1)}(v,w)\\
    F_{(1,1,1)}(g_2\cdot (v,w)) &=  - F_{(1,1,1)}(v,w)\\
  \end{align*}

  Additionally, we can choose a linear $G$-action on $\bbR^4$ that is consistent with the previous equations. In particular, if we define:

  \begin{align*}
    g_1\cdot (x, y, z, u) &= (x, -z, y, -u)\\
    g_2 \cdot (x, y, z, u) &= (-x, -y, -z, -u)
  \end{align*}

  Then it is easy to check that the alternating sum map $\Psi: S^1\times S^3 \rightarrow \bbR^4$

  \[
    (v, w) \mapsto \left(F_{(0,0,1)}(v,w), F_{(0,1,1)}(v,w), F_{(1,0,1)}(v,w), F_{(1,1,1)}(v,w)\right)
  \]

  is actually a $G$-equivariant map whose zeros are exactly the configurations of planes that equipartition the measure and have the desired intersection property.
  \vskip .5em
  \textbf{Step 2}
  \vskip .5em

  Suppose now by contradiction that $\Psi$ does not have a zero. This means that $\overline{\Psi}: S^1\times S^3 \rightarrow S^3$ is a well defined
  $G$-equivariant map.

  Denote by $\Psi_a$, $a\in S^1$, the map $\Psi_a: S^3 \rightarrow S^3$, $\Psi_a(p) = \overline{\Psi}(a, p)$; this function have two key properties:
  \begin{enumerate}
  \item $\forall a \in S^1$, $\Psi_a$ is antipodal
  \item $\forall a, b \in S^1$, $\Psi_a$ and $\Psi_b$ are homotopic
  \end{enumerate}

  However, the map induced by $g_1$ on the sphere has degree $-1$ and thus we have:
  \[
    [\Psi_a] = [\Psi_{g_1\cdot a}] = [g_1\cdot \Psi_a] = - [\Psi_a]
  \]

  Thus $\Psi_a$ is null-homotopic, but Borsuk-Ulam implies that an antipodal map of $S^3$ can't be. $\lightning$

\end{proof}

\section*{Topology facts needed in the proof}

\begin{theorem}
  If $A\in O(n)$ is an orthogonal matrix, then the induced continuous map $A:S^{n-1} \rightarrow S^{n-1}$ has degree $\deg(A) = \det(A)$.
\end{theorem}
\begin{proof}
  Assume $\det(A) = 1$ (i.e. $A\in SO(n)$).

  Let $P$ an invertible matrix that puts $A$ in Jordan normal form ($R_{\theta}$ denotes the $2\times 2$ matrix of the rotation by $\theta$):

  \[
    A = P^{-1}\matrice{
      R_{\theta_1} &            & & &\\
                 & R_{\theta_2} & & &\\
                 &            & \dots & &\\
                 &            &       & R_{\theta_k} & \\
                 &            &       &           & Id_{n-2k}
    } P
  \]

  Then, there is a path $\gamma: Id\rightsquigarrow A$ defined as:

   \[
    \gamma(t) = P^{-1}\matrice{
      R_{t\theta_1} &            & & &\\
                 & R_{t\theta_2} & & &\\
                 &            & \dots & &\\
                 &            &       & R_{t\theta_k} & \\
                 &            &       &           & Id_{n-2k}
    } P
  \]

  As a result, the map $A:S^{n-1} \rightarrow S^{n-1}$ is homotopic to the identity through the homotopy $H:S^{n-1} \times \left[0,1\right]\rightarrow S^{n-1}$;
  $H(x, t) = \gamma(1-t)x$.

  Hence $deg(A) = deg(Id) = 1$.

  \vskip .5em


  Let now $\det(A) = -1$, then $QA \in SO(n)$ where
  \[
    Q := \matrice{
      Id & \\
    & -1}
  \]

  this means that $1=deg(QA) = deg(Q)deg(A) = -deg(A)$.
\end{proof}

\begin{theorem}
  Let $f:S^{n-1} \rightarrow S^{n-1}$ be an antipodal map. Then $\deg(f) \neq 0$.
\end{theorem}
\begin{proof}
  Suppose by contradiction $\exists f:S^{n-1} \rightarrow S^{n-1}$ antipodal and $\deg(f)=0$.

  Then $f$ can be extended to a map $F:D^n \rightarrow S^{n-1}$; using this map it is possible to construct $\tilde F: S^n\rightarrow S^{n-1}$:
  \[
    \tilde F(x_1, \dots, x_{n+1})=
    \begin{cases}
      F(x_1, \dots, x_n) & \mbox{if }x_{n+1} \geq 0\\
      -F(-x_1, \dots, -x_n) & \mbox{if }x_{n+1} \leq 0
    \end{cases}
  \]

  It is well defined because on the intersection of the two pathes (i.e. the horizontal equator) both sides coincide with $f$; what is more, $\tilde F$ is antipodal:
  \[
    \tilde F(-(x, x_{n+1})) = -F(-(-x)) = -F(x) = - \tilde F((x, x_{n+1}))
  \]
  \[
    \tilde F(-(x, x_{n+1})) = F(-x) = - (-F(-x)) = - \tilde F((x, x_{n+1}))
  \]

  Thus $\tilde F$ violates Borsuk-Ulam.

\end{proof}
\end{document}
