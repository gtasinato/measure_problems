\documentclass[9pt]{article}

\usepackage{amsmath}
\usepackage{amssymb}
\usepackage{amsthm}
\usepackage[english]{babel}
\usepackage{graphicx} %figures 
\usepackage{csquotes} %recommended package with bib LaTeX
% \usepackage[style=alphabetic, sorting=ynt]{biblatex} 
\usepackage{faktor} %nicer quotient 
\usepackage{tabularx} %fancy tables
\usepackage{stmaryrd} % lightning symbol 
\usepackage{subcaption} % multiple figures in the same picture
\usepackage{hyperref} % nicer hyperlinks in the text
\usepackage{bbm} %mathbbm 1 command
\usepackage{enumerate} %stylizing enumerate

\newcommand{\bbZ}{\mathbb{Z}}
\newcommand{\bbR}{\mathbb{R}}
\newcommand{\bbF}{\mathbb{F}}
\newcommand{\bbS}{\mathbb{S}}

\newcommand{\ccH}{\mathcal{H}}
\newcommand{\ccO}{\mathcal{O}}

\newcommand{\hmaps}[3][\bbZ_2]{\left[#2, #3\right]_{#1}}
\newcommand{\ztwo}{\bbZ_2}
\newcommand{\hsimp}[3][2]{\mbox{hSimp}_{#1}\left(#2, #3\right)}
\newcommand{\simp}[3][2]{\mbox{Simp}_{#1}\left(#2, #3\right)}
\newcommand{\rar}{\rightarrow}

\newcommand{\scalar}[2]{\left( #1 \vert #2\right)}
\newcommand{\matrice}[1]{\begin{pmatrix}#1\end{pmatrix}}
\newcommand{\norm}[1]{ \lVert #1 \rVert}
\newcommand{\gen}[1]{ \left\langle #1 \right\rangle}

\newtheorem{theorem}{Theorem}
\newtheorem{lemma}{Lemma}
\newtheorem{question}{Question}
\newtheorem{definition}{Definition}

%change \varepsilon to be he default choice
\let\uglyepsilon\epsilon
\let\epsilon\varepsilon


\usepackage{biblatex}
%\bibliographystyle{plain} % we choose the "plain" reference style
\bibliography{biblio} % Entries are in the refs.bib file



\title{Splitting lines in $\bbR^3$}
\date{}
\author{}
\begin{document}
\maketitle
\section*{Notation}
\begin{itemize}
\item $B_r(p)$ is the (closed) ball centered in $p$ of radius $r$.
\item $X := S^3\times S^3 \times S^3$ as a subset of polynomials $\bbR[t_1, t_2, t_3]$ (product of three affine polynomials).
\item $X$ is a metric space with the distance:
  \[
    d(x,y) = \max_{i\in I}\vert x_i - y_i\vert
  \]
  i.e. the distance between two polynomials is the biggest difference in their coefficients (it is always finite since the set of possible multi-indices is finite).
  
    \item Given $x\in X$, $Z(x) := \{p\in \bbR^3 \vert x(p) = 0\}$ is the plane configuration induced by $x$
    \item Given $x\in X$ and $\omega \in G$, the orthant $\ccO(x, \omega)$ is the (potentially empty) open set in $\bbR^3 \setminus Z(x)$ defined as:
    \[
        \ccO(x, \omega) := \{p\in \bbR^3\vert (-1)^{\omega_i}x_i(p)>0\ \forall i\}
    \]

    \item The degenerate set in $X$ is
     $A :=\left\{x\in X \vert \mbox{at least one orthant is empty}\right\}$
    \item $G_\pm :=  \left(\ztwo\times \ztwo \times \ztwo\right) \rtimes \bbS_3$ where $\bbS_3$ is the permutation group on three elements
    \item $G := \ztwo\times \ztwo \times \ztwo$
    \item $\Gamma$ is a collection of $n$ different (affine) lines in $\bbR^3$
    \item Given $\epsilon>0$, the cut-off function $\eta_\epsilon: \bbR \rightarrow \bbR$ is:
    \[
        \eta_\epsilon(x)= \begin{cases}
            0 & \mbox{ if }x\leq \epsilon\\
            \frac{1}{\epsilon}x -1 & \mbox{ if }\epsilon\leq x \leq 2\epsilon\\
            1 & \mbox{ if } x\geq 2\epsilon
        \end{cases}
      \]
\end{itemize}

\section{Preamble: $G_\pm$-representations}
\label{sec:preamble}
Before starting it is useful to clarify how $G_\pm$ acts on the target space (either $\bbR^G$ or $\bbR^{G\setminus 0}$).

The group acts on the two vector spaces differently. On the first space $V := \bbR^G$, the action is a permutation of the coordinates according to the following rule:

\[
  ((\alpha\rtimes \sigma)\cdot v)_{\omega} = v_{\alpha + \sigma^{-1}\cdot \omega}
\]

\vskip 1em

On the other end, the action on the space $W := \bbR^{G\setminus 0}$ is given by:

\[
  ((\alpha\rtimes \sigma)\cdot v)_{\omega} = (-1)^{\scalar{\alpha}{\sigma^{-1}\cdot \omega}} v_{\sigma^{-1}\cdot \omega}
\]

\vskip 1em

There is a $G_\pm$-linear map between the two representations, the alternate sum map, $T:V\rightarrow W$ defined as:

\[
  \left(Tv\right)_\omega = \sum_{\alpha\in G} (-1)^{\scalar{\alpha}{\omega}} v_\alpha
\]

It is an easy computation to see that $\ker(T)$ is the linear space generated by $\mathbbm{1}$ (the vector of all $1$s).

\section{Proof}
The goal of this notes is to prove the following fact:
\begin{theorem}
    Suppose $\Gamma$ is not degenerate (i.e. $\nexists x \in X$ such that $\bigcup \Gamma \subseteq Z(x)$). Then $\exists x\in X$ such that every orthant
    intersect at most $\frac{n}{2}$ lines in $\Gamma$.
\end{theorem}

Recall that the Guth function of parameter $\delta>0$ is defined as $I_\delta: X\rightarrow V$:
\[
    \left(I_\delta(x)\right)_\omega = \sum_{\gamma \in \Gamma} \eta_\epsilon\left(\int_{N_\delta\gamma \cap \ccO(x, \omega) \cap B_R} \eta_\epsilon(\vert x(p) \vert) \delta^{-3} d p \right)
\]

Where $\epsilon := \epsilon(\delta)$ and $R := R(\delta)$ are suitable functions (see \cite{GUTH2015}) 
Since the Guth functions point-wise converge to the counting function for intersections, if we show that, for all sufficiently small $\delta>0$,  $T\circ I_\delta$  has a zero we obtain that $\exists x \in X$ for which the counting function is multiple (up to integer rounding) of $\mathbbm{1}$; by intersection counting the correct multiple is $\frac{n}{2}$. 

\vskip .5em
The first part of the proof will be dedicated to show that for a small enough $\delta$ the Guth function $I_\delta$ is $G_\pm$-homotopic to the induced function $f$ for a measure, hence $TI_\delta$ and $Tf$ are $G_\pm$-homotopic.
What is more, we will show that this holds on a manifold with boundary $E\subseteq X\setminus A$ of dimension $7$; since $E$ avoids $A$, the action of $G_\pm$ is free on $E$. 

The zero set of the homotopy is going to be a free $G_\pm$-manifold of dimension $1$ with (non-empty) boundary in $E\times I$ and we will show that it has to avoid $\partial E\times I$. By choosing a clever measure we get that there has to be exactly one orbit of zeros on the final level of the homotopy. By the fact that $G_\pm$-homotopies can change the number of zeros only by multiples of $|G_\pm|$, the function $TI_\delta$ has to have at least one zero on $E$ as desired.

\subsection{Find the $\delta$}

The goal of this section is to find the suitable $\delta >0$ as previously mentioned.
\begin{definition}
    Given $x\in X$ and $\gamma\in \Gamma$, $\delta >0$ is acceptable for $x$ with witness $p\in \bbR^3$ if:
    \begin{itemize}
        \item $p\in \gamma$ and $(1+\delta) \norm{p} < R(\delta)$ (i.e. $B_\delta(p) \subseteq int(B_R)$)
        \item $\min_{q\in B_\delta(p)} \vert x(q)\vert > 2\epsilon$
    \end{itemize}
\end{definition}

The first useful remark is that $\forall x\in X$  there is $\delta_x>0$ admissible. In fact, fix $x\in X$, then there is $\gamma \in \Gamma$ such that $\gamma$
is not contained in $Z(x)$ thus it is possible to pick $p\in \gamma\setminus Z(x)$. 
Since $R\rightarrow \infty$, $\epsilon\rightarrow 0$ and $\min_{q\in B_\delta(p)} \vert x(q)\vert \rightarrow \vert x(p)\vert >0$ with $\delta \rightarrow 0$
eventually there will be a $\delta_x >0$ for which all the conditions will be satisfied simultaneously $\Rightarrow$ $\delta_x$ is acceptable.

\begin{lemma}
    There is $\tilde \delta >0$ such that $\tilde\delta$ is acceptable $\forall x \in X$.
\end{lemma}
\begin{proof}
    $\forall \delta >0$ define $U(\delta) := \left\{ x\in X \vert \delta \mbox{ is acceptable for }x\right\}$. 
    Since $X = \bigcup_{\delta >0} U_\delta$ and $U_\alpha \subseteq U_\beta$ whenever $\alpha \leq \beta$, in order to obtain the thesis it is sufficient by compactness
    to show that $U_\delta$ is open $\forall \delta$.

    Fix $x\in U_\delta$ and $p$ a witness. Then it is enough to show:

    \noindent \textbf{Claim:} If $y$ is close enough to $x$, $\delta$ is acceptable for $y$ with witness $p$.
    \vskip .5em
    \noindent \textbf{Proof:} The only condition we need to verify is that $\min_{q\in B_\delta(p)} \vert y(q)\vert >2\epsilon$.

    $\forall q\in B_\delta(p)$, we have that:
    \[
        \vert y(q) \vert \geq \left\vert \vert x(q) \vert - \vert x(q) - y(q)\vert \right\vert
    \]

    Since $m:y \mapsto \max_{q\in B_\delta(p)} \vert x(q) - y(q)\vert $ is continuous (lemma \ref{continuity}), it is possible to pick $y$ close enough to $x$ so that
    $m(y)< \frac{1}{2}\left(\max_{q\in B_\delta(p)} \vert x(q)\vert - 2\epsilon\right)$; thus $\vert y(q)\vert > 2\epsilon$
    $\ \forall q\in B_\delta(p)$ as desired.
\end{proof}

From now on, fix $\delta$ to be a value $\tilde \delta \geq \delta > 0$.

\subsection{Find the homotopy}
\label{sub:homotopy}

Since the space of bounded Borel measures on $\bbR^3$ is convex, the convex combination of any two measure gives a $G_\pm$-homotopy between the two induced functions. What is more, such an homotopy is never $0$ on the degenerate set $A$ by construction; hence, if we can construct an homotopy between the Guth function and a measure avoiding zeros on $A$ then we can do the same with any other measure.

The key observation is that the Guth function is already almost an induced function of a measure, the only difference is the cut-off function wrapping the integral and the function 
under the integral sign; hence the strategy is to construct homotopy for these two functions that respect the properties we are interested in at all times.

Define thus the two homotopies:

\[
    \alpha_t(x) = t + (1-t)\eta_\epsilon(x)
\]
\[
    \beta_t(x) = t x + (1-t)\eta_\epsilon(x)
\]

Finally, we can define the combined homotopy; that is, on the coordinate $\omega$:

\[
    \left(H_t(x)\right)_\omega := \sum_{\gamma \in \Gamma} \beta_t \left(\int_{N_\delta\gamma \cap \ccO(x, \omega) \cap B_R} \alpha_t (\vert x(p) \vert) \delta^{-3} d p \right)
\]

This is clearly $G_\pm$-equivariant, $H_0 \equiv I_\delta$, $H_1 \equiv \int_{N_\delta\Gamma \cap \ccO(x, \omega) \cap B_R} \delta^{-3} d p$
 (the induced function for the measure supported on $N_\delta\Gamma \cap B_R$); the only property left to check is that it is never a multiple of the $\mathbbm{1}$ vector 
 on degenerate configurations.

 Fix $x \in A$. Then, there is $\omega_x \in G$ such that $\ccO(x, \omega_x)$ is empty $\Rightarrow$ $H_t(x)_{\omega_x} \equiv 0$ at every $t$. 
 
 \vskip .5em 
 \noindent \textbf{Claim:} There is an element $\xi$ and $\uglyepsilon$ such that $H_t(x)_\xi \geq \uglyepsilon >0$ at all times.

 \vskip .3em
 Since the argument of $\beta_t$ is always positive, $H_t$ is a sum of positive functions hence it is enough to prove that $\exists \gamma \in \Gamma$ and $\xi \in G$
 such that
 \[
    \beta_t \left(\int_{N_\delta\gamma \cap \ccO(x, \xi) \cap B_R} \alpha_t (\vert x(p) \vert) \delta^{-3} d p \right) \geq \uglyepsilon >0 
\] 

Let $\tilde p\in \gamma$ ($\gamma \in \Gamma$) be a witness for the acceptability of $\delta$ and let $\xi$ be the index of the orthant containing $\tilde p$. Then, $\forall t\in [0,1]$:
\[
    \min_{p\in B_\delta(\tilde p)}\vert x(p) \vert >2\epsilon \Rightarrow \alpha_t(\vert x(p)\vert ) = 1
\]

Hence:
\[
    \int_{N_\delta\gamma \cap \ccO(x, \xi) \cap B_R} \alpha_t (\vert x(p) \vert) \delta^{-3} d p \geq \int_{B_\delta(\tilde p)} \delta^{-3} d p = \frac{4\pi}{3}
\]

Since $\beta_t(x)$ is monotone increasing in $t$ if $x\geq 1$
\[
    \beta_t(\int_{N_\delta\gamma \cap \ccO(x, \xi) \cap B_R} \alpha_t (\vert x(p) \vert) \delta^{-3} d p) \geq \eta_\epsilon\left(\frac{4\pi}{3}\right) = 1
  \]

  As a result, the map $T\circ H_t$ is a $G_\pm$-homotopy between the test maps that doesn't have any zeros on $A$. Without loss of generality we can assume that $T\circ H_t$ is an homotopy between $I_\delta$ and the induced function of the target measure defined in lemma \ref{measure}.

\subsection{Find $E$ - Conclusion}
\label{sec:conclusion}

By compactness of $X\times I$, there is an $\epsilon > 0$ such that $\forall \alpha \in G \setminus 0$,  $\left(T\circ H_t\right|_A)_\alpha  > \epsilon$. It is thus possible to choose a small $G_\pm$-invariant tubolar neighborhood $N_A$ such that  $\left(T\circ H_t\right|_{N_A})_\alpha  > \frac{2\epsilon}{3}$. For convenience, choose $N_A$ to be closed and denote by $U_A$ its interior.
  
\vskip .5em
Let $P := \{ x\in X \vert \mbox{ one of the planes is parallel to the plane }t_1 = 0\}$. By construction $ P = \left(S^1\times S^3\times S^3\right)\times \left(S^3\times S^1\times S^3\right)\times \left(S^3\times S^3\times S^1\right)$ and thus it is not a manifold.

The corner points are contained in $A$, hence $E = P\setminus U_A$ is a $7$-dimensional manifold with boundary on which the action of $G_\pm$ is free.
\vskip .75em
By choosing suitable small compatible $G_\pm$-triangulations for $X$, $N_A$, $[0,1]$ and $E$, we there is a $G_\pm$-map  $f : X\times [0,1] \rightarrow W$ that is:

\begin{itemize}
\item $f$ is affine on every simplex of $X\times [0,1]$
\item $G_\pm$-homotopic to $T\circ H$
\item very close to $T\circ H$ (e.g. $\max_{x\in X, t\in I} \norm{f(x,t)-TH_t(x)}\leq \frac{\epsilon}{3}$)
\item There is a unique orbit $G_\pm x$ in $X$ such that $f(x, 1) = 0$ and $x\neq A$
\item $f|_E$ is generic (i.e. $f^{-1}(0)$ intersect only faces of dimension at least $7 = \vert G\vert -1$)
\item $\norm{f|_{\partial E}}\geq \frac{\epsilon}{3}$
\end{itemize}

(see lemma \ref{generic} for a proof).

As a result, $Z:=f^{-1}(0)\cap E\times I$ is a $1$-dimensional PL-manifold with boundary that is $G_\pm$ invariant.

If we choose a connected component starting from one of the point in $Z\cap E\times \{1\}$, this is an interval with exactly one endpoint on $E\times \{1\}$ and does not intersect $\partial E\times I$. It follows that the other endpoint has to be on $E\times \{0\}$ and thus we showed that $|Z\cap E\times \{0\}| = 1 \mod |G_\pm|$, hence non zero. Since this quantity is preserved under $G_\pm$-homotopies, the same has to be true for $T\circ H_0 = T\circ I_\delta$ as desired.

\section{Technical Lemmas}
\label{sec:lemmas}

\begin{lemma}[convergence in $X$ implies global ptwise convergence]\label{convergence}
 Let $x_n \rightarrow x_\infty$ a converging sequence in $X$ with the distance previously defined. Then $\forall p\in \bbR^3$, $x_n(p) \rightarrow x_\infty(p)$.
\end{lemma}
\begin{proof}
  Fix $\epsilon > 0$ and $p\in \bbR^3$ and denote by $y^i$ the homogeneous component of degree $i$ of a polynomial $y\in X$. If $p=0$ then, for $n$ big enough:
  \[
    \vert x_n(0) - x_\infty(0) \vert = \vert x_n^0 - x_\infty^0 \vert \leq d(x_n, x_\infty) \leq \epsilon
\]

Thus we can assume $\norm{p} \neq 0$ and denote by $\hat{p}:=\frac{p}{\norm{p}}$.

  If $0<\norm{p} \leq 1$; for $n$ big enough we have: 
  \[
    \vert x_n(p) - x_\infty(p) \vert \leq \sum_{i=0}^3 \norm{p}^i\vert x_n^i(\hat{p})-x_\infty^i(\hat{p})\vert \leq \sum_{i=0}^3c_id(x_n, x_\infty)\leq C d(x_n, x_\infty)\leq \epsilon
  \]

  for some constants $c_i, C$.

  Analogously, if $1\leq \norm{p}$, then:
  \[
    \vert x_n(p) - x_\infty(p) \vert \leq \sum_{i=0}^3 \norm{p}^i\vert x_n^i(\hat{p})-x_\infty^i(\hat{p})\vert \leq \sum_{i=0}^3\norm{p}^ic_id(x_n, x_\infty)\leq \norm{p}^3 C d(x_n, x_\infty)
  \]

  However, since $p$ is fixed, we can choose $n$ big enough such that
  \[
    d(x_n, x_\infty)\leq \frac{\epsilon}{\norm{p}^3C}
\]

hence the sequence $x_n(p)$ converges to $x_\infty(p)$ as desired.
\end{proof}

\begin{lemma}[max function is continuous]
  \label{continuity}
  Fix $x\in X$, $\delta >0$ and $p\in \bbR^3$, then the function $m:X\rightarrow \bbR$, $m(y) = \max_{q\in B_\delta(p)} \vert y(q)-x(q)\vert$ is continuous.
\end{lemma}
\begin{proof}
  It is enough to prove sequential continuity. Let $y_n\rightarrow y_\infty$ be a converging sequence in $X$ $\Rightarrow$ $y_n(p) \rightarrow y_\infty(p)$ (lemma \ref{convergence}).

  Let $q_n$ be a point that realizes $m(y_n)$ (i.e. $m(y_n) = \vert y_n(q_n)-x(q_n)\vert$), up to taking a sub-sequence we can assume $q_n$ converges to some point $q_\infty\in B_\delta(p)$.

  
  What is more, the family $\{y_n\}$ is equicontinuous as functions $y_n:B_\delta(p) \rightarrow \bbR$ (they are differentiable and have bounded derivative) and thus they converge uniformly on $B_\delta(p)$.
  
  \vskip 1em
  \textbf{CLAIM: } $q_\infty$ realizes $m(y_\infty)$.

  Assuming the claim, we get that for $n$ big enough:

  \begin{align*}
    \vert m(y_n) - m(y_\infty)\vert &= \vert y_n(q_n)-x(q_n) - y_\infty(q_\infty) + x(q_\infty)\vert \\
                                    &\leq \vert y_n(q_n) - y_n(q_\infty) \vert + \vert y_n(q_\infty) - y_\infty(q_\infty) \vert + \vert x(q_\infty) - x(q_n) \vert\\
                                    &\leq \epsilon + \epsilon + \epsilon\\
  \end{align*}

  where the last inequality holds by equicontinuity (first term), pointwise convergence (second term) and continuity of $x$ (third term).

  The only thing left to prove is the claim.

  \vskip .5em
  
  
\end{proof}
\begin{lemma}[Target Measure]
  \label{measure}
  Let $\mu$ the probability measure with support on $S=\{(t, t^2, t^3)\in \bbR^3 | t\in [-1,1]\}$ and uniform density. Then there is a unique $G_\pm$-orbit of points in $P$ that equipartitions $\mu$.

  Equivalently, up to order and signs there is a unique triple of planes such that every orthant has the same measure and the first one is parallel to the horizontal plane $\{p\in \bbR^3 \vert p_1 = 0\}$. 
  
\end{lemma}

\begin{lemma}[Generic Homotopy]
  \label{generic}

  Let $TH_t: X \rightarrow W$ the homotopy constructed in section \ref{sub:homotopy} and $\epsilon > 0$ small enough. Then there are compatible $G_\pm$-triangulation for $X$, $E$, $I:=[0,1]$, $X\times I$ and a $G_\pm$-function $f:X\times I\rightarrow W$ with the following properties:
  \begin{enumerate}
  \item $\max_{(x,t)\in X\times I} \norm{f(x, t) - TH_t(x)} < \epsilon$
  \item $f$ is $G_\pm$-homotopic to $TH$
  \item $f$ is affine on every simplex in $X\times I$
  \item $f$ is generic: i.e. $f^{-1}(0)$ intersects only simplices of dimension at least $7=\dim W$
  \item $f(x,1) = 0$ on exactly one orbit in $E$ 
  \end{enumerate}
\end{lemma}

\begin{proof}
  By equivariant simplicial approximation we can find small enough triangulations for the spaces and an affine map $g$ that is $G_\pm$-homotopic to $TH$ and $\max_{(x,t)\in X\times I} \norm{f(x, t) - TH_t(x)} < \frac{\epsilon}{2}$.
  By lemma \ref{smooth}, the property of having exactly one orbit of zeros is preserved under perturbations small enough (since $0$ is a regular value of the function on $E$) hence the only condition we need to show is that we can find a perturbation that is generic.
\end{proof}

\begin{lemma}\label{smooth}
  Fix $\mu$ the measure defined in lemma \ref{measure} and denote by $g:X\rightarrow V$ the function $g(x)_\omega = \int_{\ccO(x, \omega)}\mu$, then $g$ is smooth and its critical values are away from the diagonal of $V$ (i.e. $0$ is a regular value for $Tg$).
\end{lemma}

\begin{proof}
  
  [Sketch / find a less convoluted proof for last point in lemma \ref{generic}]
  
  % By construction of the measure, the value of $g(x_1, x_2, x_3)_\omega$ is the sum of the length of some sub-intervals of $[-1, 1]$ with endpoints in the set
  % \[
  %   \{t_\alpha | x_1(t_\alpha,t_\alpha^2, t_\alpha^3)x_2(t_\alpha,t_\alpha^2, t_\alpha^3)x_3(t_\alpha,t_\alpha^2, t_\alpha^3)=0\}
  % \]

  % For convenience, denote by $X(t)$ the polynomial $ x_1(t,t^2, t^3)x_2(t,t^2, t^3)x_3(t,t^2, t^3)$.

  % By the implicit function theorem, if $x\in E$ these are locally smooth functions $t_\alpha$ hence the function $g$ is smooth on $E$ and the differential is (some sum) of the differentials of $t_\alpha$ that is invertible if $Tg(x)=0$\footnote{I need to write down the matrices and check that I made no sign mistakes but it should be fine. I assume there is a more straightforward proof of the fact that the induced function by $\mu$ cross the diagonal transversely but this was the only one I could come up with.}.


  % \hrule

  \vskip .5em

  By lemma \ref{measure}, $(Tg)^{-1}(0) = G_\pm \bar{x}$ where
  \[
    \bar{x} = \left( (1,0,0,0), (-\frac{5}{16}, \frac{1}{2}, 1, -\frac{3}{32}), (-\frac{5}{16}, -\frac{1}{2}, 1, \frac{3}{32})\right)
  \]

  Since $G_\pm$ acts on the differential as multiplication with an invertible matrix, $0$ is a regular value for $Tg$ if and only if $Dg_x$ is full rank.


  Since $E\subseteq (\bbR^4)^3$, the $0$-set of the function $E\times \bbR \rightarrow \bbR$ defined as
  \[
    \phi(x_1,x_2,x_3, t) = x_1( t, t^2, t^3 )x_2(t,t^2,t^3)x_3(t,t^2,t^3)
  \]

  gives a parametrization of the intersections of points of $E$ with the moment curve in a small neighborhood of $\bar{x}$ by the implicit function theorem. Denote by $z_i$ the local function 

  
  The differential $Dg_x$ is the matrix $8\times 12$ matrix\footnote{The entries will be indexed by $G$ on the rows.}: 
  \[
    (Dg_x)^t = \matrice{
      \partial_0 g_{(0,0,0)} & \partial_0 g_{(0,0,1)} & \partial_0 g_{(0,1,0)} & \dots & \partial_0 g_{(1,1,0)} &\partial_0 g_{(1,1,1)} \\
      & & & \ddots&&\\
      
      \partial_{11} g_{(0,0,0)} & \partial_{11} g_{(0,0,1)} & \partial_{11} g_{(0,1,0)} & \dots &  \partial_{11} g_{(1,1,0)} & \partial_{11} g_{(1,1,1)} \\
      }
    \]

    
\end{proof}
\cite{guth}
test double test
\end{document}
% LocalWords:  equicontinuity equivariant simplicial homotopy orthant
% LocalWords:  homotopic equipartitions simplices triangulations Guth
% LocalWords:  invertible equicontinuous homotopies Borel
% LocalWords:  parametrization
